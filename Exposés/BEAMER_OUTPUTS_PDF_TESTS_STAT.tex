% Options for packages loaded elsewhere
\PassOptionsToPackage{unicode}{hyperref}
\PassOptionsToPackage{hyphens}{url}
%
\documentclass[
]{article}
\usepackage{amsmath,amssymb}
\usepackage{iftex}
\ifPDFTeX
  \usepackage[T1]{fontenc}
  \usepackage[utf8]{inputenc}
  \usepackage{textcomp} % provide euro and other symbols
\else % if luatex or xetex
  \usepackage{unicode-math} % this also loads fontspec
  \defaultfontfeatures{Scale=MatchLowercase}
  \defaultfontfeatures[\rmfamily]{Ligatures=TeX,Scale=1}
\fi
\usepackage{lmodern}
\ifPDFTeX\else
  % xetex/luatex font selection
\fi
% Use upquote if available, for straight quotes in verbatim environments
\IfFileExists{upquote.sty}{\usepackage{upquote}}{}
\IfFileExists{microtype.sty}{% use microtype if available
  \usepackage[]{microtype}
  \UseMicrotypeSet[protrusion]{basicmath} % disable protrusion for tt fonts
}{}
\makeatletter
\@ifundefined{KOMAClassName}{% if non-KOMA class
  \IfFileExists{parskip.sty}{%
    \usepackage{parskip}
  }{% else
    \setlength{\parindent}{0pt}
    \setlength{\parskip}{6pt plus 2pt minus 1pt}}
}{% if KOMA class
  \KOMAoptions{parskip=half}}
\makeatother
\usepackage{xcolor}
\usepackage[margin=1in]{geometry}
\usepackage{graphicx}
\makeatletter
\def\maxwidth{\ifdim\Gin@nat@width>\linewidth\linewidth\else\Gin@nat@width\fi}
\def\maxheight{\ifdim\Gin@nat@height>\textheight\textheight\else\Gin@nat@height\fi}
\makeatother
% Scale images if necessary, so that they will not overflow the page
% margins by default, and it is still possible to overwrite the defaults
% using explicit options in \includegraphics[width, height, ...]{}
\setkeys{Gin}{width=\maxwidth,height=\maxheight,keepaspectratio}
% Set default figure placement to htbp
\makeatletter
\def\fps@figure{htbp}
\makeatother
\setlength{\emergencystretch}{3em} % prevent overfull lines
\providecommand{\tightlist}{%
  \setlength{\itemsep}{0pt}\setlength{\parskip}{0pt}}
\setcounter{secnumdepth}{-\maxdimen} % remove section numbering
\setbeamercolor{structure}{fg=green!60!black}
\setbeamercolor{palette primary}{use=structure,fg=white,bg=green!60!black}
\setbeamercolor{palette secondary}{use=structure,fg=white,bg=green!40!black}
\setbeamercolor{frametitle}{bg=green!60!black,fg=white}
\setbeamercolor{title}{bg=white,fg=green!60!black}
\setbeamercolor{subtitle}{fg=green!60!black}
\setbeamercolor{author}{fg=green!60!black}
\ifLuaTeX
  \usepackage{selnolig}  % disable illegal ligatures
\fi
\usepackage{bookmark}
\IfFileExists{xurl.sty}{\usepackage{xurl}}{} % add URL line breaks if available
\urlstyle{same}
\hypersetup{
  pdftitle={Consommation alimentaire par zone rurale et urbaine (SEN2018)},
  pdfauthor={RIRADJIM NGARMOUNDOU Trésor},
  hidelinks,
  pdfcreator={LaTeX via pandoc}}

\title{Consommation alimentaire par zone rurale et urbaine (SEN2018)}
\author{RIRADJIM NGARMOUNDOU Trésor}
\date{15 May 2025}

\begin{document}
\maketitle

\texttt{\{r\ knitr::opts\_chunk\$set(echo\ =\ TRUE,\ warning\ =\ FALSE,\ message\ =\ FALSE)\ library(ggplot2)\ library(dplyr)\ library(car)\ library(rstatix)\ library(kSamples)\ df\ \textless{}-\ readRDS(\textquotesingle{}../data/df\_clean.rds\textquotesingle{})}

\subsection{1. Contexte}\label{contexte}

L'EHCVM 2018 vise à harmoniser les statistiques de conditions de vie
dans l'UEMOA.

\subsection{2. Objectifs}\label{objectifs}

\begin{itemize}
\tightlist
\item
  Données comparables sur les conditions de vie
\item
  Mesure harmonisée de la pauvreté
\item
  Suivi des ODD
\item
  Aide à la politique publique
\end{itemize}

\subsection{3. Variables clés}\label{variables-cluxe9s}

\begin{itemize}
\tightlist
\item
  \textbf{zone} : Urbaine vs Rurale
\item
  \textbf{conso\_total} : consommation alimentaire totale
\end{itemize}

\subsection{4. Analyse descriptive}\label{analyse-descriptive}

\texttt{\{r\ \#\ Boxplot\ par\ zone\ ggplot(df,\ aes(x\ =\ zone,\ y\ =\ conso\_total))\ +\ \ \ geom\_boxplot(fill\ =\ \textquotesingle{}skyblue\textquotesingle{})\ +\ \ \ labs(title\ =\ \textquotesingle{}Consommation\ alimentaire\ par\ zone\textquotesingle{},\ x\ =\ \textquotesingle{}Zone\textquotesingle{},\ y\ =\ \textquotesingle{}conso\_total\textquotesingle{})\ \#\ Statistiques\ de\ base\ df\ \%\textgreater{}\%\ group\_by(zone)\ \%\textgreater{}\%\ \ \ summarise(n\ =\ n(),\ mean\ =\ mean(conso\_total),\ sd\ =\ sd(conso\_total))}

\subsection{5. Tests de normalité
(Shapiro-Wilk)}\label{tests-de-normalituxe9-shapiro-wilk}

\texttt{\{r\ sh\_u\ \textless{}-\ shapiro.test(df\$conso\_total{[}df\$zone==\textquotesingle{}Urbaine\textquotesingle{}{]})\ sh\_r\ \textless{}-\ shapiro.test(df\$conso\_total{[}df\$zone==\textquotesingle{}Rurale\textquotesingle{}{]})\ sh\_u;\ sh\_r}

\textbf{Résultat~:} W\_urb = 0.8977, W\_rur = 0.9796, p \textless{}
2.2e-16 -- les deux distributions ne sont pas normales.

\subsection{6. Homogénéité des variances
(Levene)}\label{homoguxe9nuxe9ituxe9-des-variances-levene}

\texttt{\{r\ res\_levene\ \textless{}-\ leveneTest(conso\_total\textasciitilde{}zone,\ data=df)\ res\_levene}

\textbf{Résultat~:} F = 30.547, p = 3.37e-08 -- variances
significativement différentes.

\subsection{7. Comparaison des médianes
(Wilcoxon)}\label{comparaison-des-muxe9dianes-wilcoxon}

\texttt{\{r\ res\_wilcox\ \textless{}-\ wilcox.test(conso\_total\textasciitilde{}zone,\ data=df)\ res\_wilcox}

\textbf{Résultat~:} W = 4638353, p \textless{} 2.2e-16 -- médianes de
conso différentes.

\subsection{8. Comparaison des distributions (KS
asympt.)}\label{comparaison-des-distributions-ks-asympt.}

\texttt{\{r\ res\_ks\ \textless{}-\ ks.test(\ \ \ df\$conso\_total{[}df\$zone==\textquotesingle{}Urbaine\textquotesingle{}{]},\ \ \ df\$conso\_total{[}df\$zone==\textquotesingle{}Rurale\textquotesingle{}{]},\ \ \ alternative=\textquotesingle{}two.sided\textquotesingle{}\ )\ res\_ks}

\textbf{Résultat~:} D = 0.2242, p \textless{} 2.2e-16 (approx.) --
distributions différentes.

\subsection{9. Taille d'effet (Cohen's d)}\label{taille-deffet-cohens-d}

\texttt{\{r\ res\_cohen\ \textless{}-\ df\ \%\textgreater{}\%\ cohens\_d(conso\_total\textasciitilde{}zone,\ var.equal=FALSE)\ res\_cohen}

\textbf{Résultat~:} d = -0.4047 (small) → consommation rurale plus
faible de 0.4~écarts-types.

\subsection{10. Test d'Anderson--Darling}\label{test-dandersondarling}

\texttt{\{r\ ad\_res\ \textless{}-\ ad.test(conso\_total\textasciitilde{}zone,\ data=df)\ ad\_res}

\textbf{Résultat~:} A2 = 187.51, p \textless{} 0.001 → distributions
globalement différentes.

\subsection{11. Test de permutation (KS
emp.)}\label{test-de-permutation-ks-emp.}

\texttt{\{r\ obs\ \textless{}-\ max(abs(ecdf(df\$conso\_total{[}df\$zone==\textquotesingle{}Urbaine\textquotesingle{}{]})(df\$conso\_total{[}df\$zone==\textquotesingle{}Rurale\textquotesingle{}{]})\ -\ \ \ \ \ \ \ \ \ \ \ \ \ \ \ ecdf(df\$conso\_total{[}df\$zone==\textquotesingle{}Rurale\textquotesingle{}{]})(df\$conso\_total{[}df\$zone==\textquotesingle{}Urbaine\textquotesingle{}{]})))\ set.seed(123)\ perm\ \textless{}-\ replicate(2000,\ \{\ \ \ z\ \textless{}-\ sample(df\$zone)\ \ \ max(abs(ecdf(df\$conso\_total{[}z==\textquotesingle{}Urbaine\textquotesingle{}{]})(df\$conso\_total{[}z==\textquotesingle{}Rurale\textquotesingle{}{]})\ -\ \ \ \ \ \ \ \ \ \ \ ecdf(df\$conso\_total{[}z==\textquotesingle{}Rurale\textquotesingle{}{]})(df\$conso\_total{[}z==\textquotesingle{}Urbaine\textquotesingle{}{]})))\ \})\ mean(perm\ \textgreater{}=\ obs)}

\textbf{Résultat~:} p\_perm = 0.3325 -- pas de différence significative
via permutation.

\subsection{12. Khi² sur classes de
conso}\label{khiuxb2-sur-classes-de-conso}

\texttt{\{r\ df2\ \textless{}-\ df\ \%\textgreater{}\%\ mutate(cat\ =\ ntile(conso\_total,3))\ tab\textless{}-table(df2\$cat,df2\$zone)\ chisq.test(tab);\ phi2\textless{}-\ chisq.test(tab)\$statistic/sum(tab);\ phi2}

\textbf{Résultat~:} X² = 374.21, p \textless{} 2.2e-16, Phi² = 0.0524
(faible) → association zone--catégorie.

\subsection{13. ANOVA \& η²}\label{anova-ux3b7uxb2}

\texttt{\{r\ aov\_res\ \textless{}-\ aov(conso\_total\textasciitilde{}zone,\ data=df)\ summary(aov\_res)\ eta\_squared(aov\_res)}

\textbf{Résultat~:} F=289.7, p\textless2e-16, η² = 0.039 → zone explique
3.9\% de la variance.

\subsection{Conclusion}\label{conclusion}

\begin{itemize}
\tightlist
\item
  Distributions non normales, variances hétérogènes.
\item
  Différences significatives de médianes et de distributions.
\item
  Effets statistiquement robustes, mais de faible ampleur.
\end{itemize}

\subsection{Limitations et
perspectives}\label{limitations-et-perspectives}

\textbf{Limitations :}

\begin{itemize}
\tightlist
\item
  \textbf{Sources de biais} : déclarations auto-rapportées, erreurs de
  saisie, valeurs manquantes.
\item
  \textbf{Données transversales} : impossibilité d'analyser l'évolution
  dans le temps.
\item
  \textbf{Granularité} des données : conso\_total agrégée peut masquer
  des dynamiques par catégorie d'aliment.
\item
  \textbf{Effets non étudiés} : taille du ménage, niveau de revenu,
  saisonnalité des achats.
\end{itemize}

\textbf{Perspectives :}

\begin{itemize}
\tightlist
\item
  \textbf{Analyse multivariée} : modèles de régression pour contrôler
  d'autres facteurs explicatifs.
\item
  \textbf{Découpage thématique} : étude par groupes d'aliments
  (céréales, protéines, etc.).
\item
  \textbf{Approche longitudinale} : intégrer plusieurs vagues d'enquête
  pour mesurer l'évolution.
\item
  \textbf{Visualisations avancées} : courbes densité, heatmaps par
  région.
\item
  \textbf{Dimension socio-économique} : incorporation des variables de
  revenu, éducation.
\end{itemize}

\end{document}
